\begin{abstract}

Exergy-based efficiencies are a measure of the thermodynamic perfection of systems and processes: they can be expressed for any plant for which the inputs and outputs are expressible in terms of exergy. However, an exact formulation of this criterion for oil and gas platforms is made difficult by the great differences in working conditions, design setups and operating strategies amongst them. In this paper, possible interpretations of exergetic efficiencies for these specific systems are presented and discussed. We assessed and compared several formulations of exergetic efficiencies introduced in the literature for relatively similar processes. They were then applied to four real-case studies of the North Sea region and their relevance was evaluated. The `input-output' exergy efficiency was above 98\% in all cases while the corresponding figures for the `consumed-produced' exergy efficiency were between 10\% and 75\%, depending on the considerations on the `consumed' and `produced' exergies and on the case study. 

\end{abstract}