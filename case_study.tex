\section{Case study}
\label{sec:comparison}

\subsection{Exergetic efficiency using different formulations}

\begin{table*}[htbp]
\scriptsize
  \centering
  \caption{Results of exergy-based efficiencies for oil and gas platforms}
    \begin{tabular*}{\linewidth}{@{\extracolsep{\fill}}lllll}
    \toprule
          & Platform A & Platform B & Platform C & Platform D \\
	\toprule
	$\varepsilon_{I,1}$ & & & & \\
	$\varepsilon_{I,2}$ & & & & \\
	$\varepsilon_{II,1}$ & & & & \\
	$\varepsilon_{II,2}$ & & & & \\
    $\varepsilon_{II,3}$ & & & & \\
	$\varepsilon_{II,4}$ & & & & \\
	$\varepsilon_{II,5}$ & & & & \\
   	\bottomrule
    \end{tabular*}%
  \label{tab:results_efficiency}%
\end{table*}%

\subsection{Discussion of the results}

\subsubsection{Product: difference in material streams}
Why we got the results we got. Negative for Platform B... Worked for Oliviras Brazilian platform and in all his subsystems. For Voldsund2010, 2012 2013 we got a positive value for the overall plaform, but in Voldsund2010 it is shown that when using this definition for subsystems, the separation system got negative value. 

\subsubsection{Fuel: also physical exergy in, Product: separation work and physical exergy out}
Why we got the results we got. Why high efficiency for Plaform B... High transit physical exergy.

\subsubsection{Fuel: also exergy decrease, Product: also exergy increase}
Why we got the results we got. Why different for molar and mass basis. Low efficiency for Platform B.

\subsection{Which formulation makes most sense?}
Leading to the component-by-component efficiency...