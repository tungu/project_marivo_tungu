
%%%%%%%%% SECTION: INTRODUCTION %%%%%%%%%

\section{Introduction}
\label{sec:introduction}
	


Conventional indicators for evaluating the performance of oil and gas platforms, such as the power consumption per unit of oil equivalent produced, or the amount of carbon dioxide produced per unit of oil equivalent, present inherent limitations. Each oil field has different natural characteristics (e.g. gas-to-oil ratio, well-fluid composition, field size) and comparing different facilities with these metrics could be misleading. They provide useful information on the energy use of the onsite processes, but they cannot be used alone to compare the performance of different facilities \cite{Svalheim2002,Svalheim2003}.

Exergy Analysis is a quantitative assessment method that is based on both the First and Second Laws of Thermodynamics. This thermodynamic method presents advantages over a conventional energy analysis: it pinpoints the locations and types of the irreversibilities taking place within a given system. As emphasised by Rivero \cite{Rivero2002}, the application of the exergy concept in the petroleum industry would provide more detailed and consistent information on the performance of petrochemical systems. The exergy concept was introduced in the literature along with the concept of exergetic efficiency, which aims at measuring the degree of thermodynamic perfection of the process under investigation.  

Formulations of exergy-based criteria of performance have been proposed from the middle of the 20$^{th}$ century, with, amongst others, the contributions of Nesselmann \cite{Nesselmann1952} and Fratzscher \cite{Fratzscher1981,Fratzscher1986}. Both works reported the definition of the exergetic efficiency of a given system as the ratio of its total exergy output to its total exergy input and discussed the advantages and drawbacks of such formulation. Grassmann \cite{Grassmann1950} and Nesselmann \cite{Nesselmann1953} suggested to define the exergetic efficiency as the ratio of the part of the exergy transfers that contribute to the transformations taking place, i.e. \emph{`consumed'} exergy, to the part of the exergy transfers that are generated within the system, i.e. \emph{`produced'} exergy. Based on these works, Baehr \cite{Baehr1965,Baehr1968} proposed his own expressions for these two terms. He also stressed the difficulty of providing a non-ambiguous definition of an exergetic efficiency, as different views on \emph{`consumed'} and \emph{`produced'} exergies may apply.

Further advances within this field include the studies of Brodyansky \cite{Brodyansky1994}, Szargut \cite{Szargut1988,Morris1994,Szargut1998}, Kotas \cite{Kotas1995} and Tsatsaronis \cite{Tsatsaronis1993,Thermoeconomics2001}. Brodyansky \cite{Brodyansky1994} suggested a systematic procedure for calculating the \emph{`produced'} and \emph{`consumed'} exergies, without regarding whether they are useful to the owner of the system. His work was based on the concept of \emph{`transit exergy'} introduced by Kostenko \cite{Kostenko1983} and discussed also in Sorin et al. \cite{Sorin1994}. Szargut \cite{Szargut1988,Morris1994,Szargut1998}, Kotas \cite{Kotas1995} and Tsatsaronis \cite{Tsatsaronis1993,Thermoeconomics2001} proposed to consider only the exergy transfers representing the \emph{`desired'} exergetic output and the \emph{`driving'} exergetic input of the system, leading to the concept of  \emph{`product'} and \emph{`fuel'} exergies. Such considerations must be consistent with the purpose of owning and operating the system of investigation \cite{Kotas1980,Kotas1980a,BejanAdrian;TsatsaronisGeorge;Moran1996,Moran1998}, both from an economic and a thermodynamic prospect. Lazzaretto and Tsatsaronis \cite{Lazzaretto1999,Lazzaretto2006} suggested a systematic procedure for defining the exergetic efficiency at a component level. However, at a process level, a unique formulation may not be available and several expressions may be appropriate \cite{Tsatsaronis1993}.

Various expressions of exergetic efficiency for separation systems have been illustrated in the literature and different considerations have applied \cite{Brodyansky1994,Kotas1995,Tsatsaronis1993,Sorin1994a}. Cornelissen \cite{Cornelissen1997} investigated three of the proposed formulations for an Air Separation Unit (ASU) and a Crude Distillation Plant (CDP). Different values were obtained, illustrating the variations and lack of uniformity across the exergetic efficiency definitions \cite{Baehr1965,Baehr1968,Lior2007}. However, the only studies dealing exclusively with oil or gas processing plants are the works of Oliveira and Van Hombeeck \cite{Oliveira1997}, Voldsund et al. \cite{Voldsund2010,Voldsund2012}, and Rian and Ertesv\aa g \cite{Rian2012}, who proposed their own interpretations of \emph{`product'} and \emph{`fuel'} exergies.

The literature seems to contain little, if none, on (i) a uniform performance parameter for oil and gas installations, (ii) the application of the exergetic efficiency concept, and on (iii) the comparison of the different definitions. This study is aimed at addressing these gaps and is part of two broader projects conducted at the Norwegian University of Science and Technology (NTNU) and at the Technical University of Denmark (DTU). Four main steps were required in the present study:
\begin{itemize}
	\item derivation and formulation of exergetic efficiencies for offshore platforms;
	\item modelling and simulation of four oil and gas facilities in the North Sea region; \textbf{(remove this one? belongs to paper 2...)}
	\item application of the different expressions;
	\item analysis of their sensitivity to changes in design and operating conditions.
\end{itemize}

This paper is structured as follows: Section \ref{sec:system_description} presents the oil and gas platforms investigated in this study and Section \ref{sec:background} describes the theoretical background. Section \ref{sec:efficiency} reports the derived definitions of exergetic efficiencies, while Section \ref{sec:comparison} illustrates the comparison of the four oil and gas facilities. The pertinence of the different exergetic efficiency formulations is discussed in Section \ref{sec:discussion} and concluding remarks are outlined in Section \ref{sec:conclusion}.
